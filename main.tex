\documentclass[t, 11pt]{beamer}
\usepackage{mathtools}
\usepackage{pdfpages}
\usepackage{graphicx}
\usepackage{tikz}
\usepackage{multimedia}
\usepackage{transparent}
\usetikzlibrary{calc}
%%%%%%%%%%%%%%%%%%%%%%%%%%%%%%%%%%%%%%%%%%%%%%%%%%%%%%%%%%%%
% absolute positioning of typeset material    
%%%%%%%%%%%%%%%%%%%%%%%%%%%%%%%%%%%%%%%%%%%%%%%%%%%%%%%%%%%%
\newcommand{\placetextbox}[4][center]{%
  % [#1]: box anchor: center (default) | 
  %                 south west | west | north west | north |
  %                 north east | east | south east | south | 
  %                 mid west | mid | mid east |
  %                 base west | base | base east 
  % #2: horizontal position (fraction of page width)
  % #3: vertical position (fraction of page height)
  % #4: content
  %
  \tikz[remember picture,overlay,x=\paperwidth,y=\paperheight]{%
    \node[anchor=#1,inner sep=0pt]
    at ($(current page.south west)+(#2,#3)$) {#4};
  }%
}
%%%%%%%%%%%%%%%%%%%%%%%%%%%%%%%%%%%%%%%%%%%%%%%%%%%%%%%%%%%%

% \usetheme{Madrid}
\usetheme{Berlin}

\def\liketitle#1{%
{\usebeamerfont{frametitle}\usebeamercolor[fg]{structure}%
\begin{flushleft}%
\vspace{-\baselineskip}% Cometic correction for space introduced by flushleft
#1\par
\end{flushleft}%
\vspace{-\baselineskip}% Cosmetic correction for space introduced by flushleft
}%
\vspace{0.75\baselineskip}%
}

\date{January 16th, 2020}
\author{Donovan Webb}
\institute{University of Bath/eBIC}
\titlegraphic{\includegraphics[width=3cm]{Diamond.png}}
\title{Placement Visit}

\AtBeginSection[]{
  \begin{frame}
  \vfill
  \centering
  \begin{beamercolorbox}[sep=8pt,center,shadow=false,rounded=false]{title}
    \usebeamerfont{title}\insertsectionhead\par%
  \end{beamercolorbox}
  \vfill
  \end{frame}
}

\begin{document}

\setbeamertemplate{navigation symbols}{}
\begin{frame}[plain]
  \maketitle
\end{frame}
\addtocounter{framenumber}{-1} % Don't count title slide

\begin{frame}{Table of Contents}
  \tableofcontents[sectionstyle=show/show, hideallsubsections]
\end{frame}

\section{What is Cryo-EM?}
\subsection{}
\begin{frame}[blank]
  \centering\vspace{1em}\Huge \[\underbracket{\text{Cryo}}_{\text{\large In Ice}}\text{-}\underbracket{\vphantom{\text{y}}\text{Electron}}_{\text{\parbox{4em}{\centering\normalsize Electrons can have shorter wavelength than light}}} \underbracket{\text{Microscopy}}_{\text{\large Its a Microscope!}} \]
\end{frame}

\begin{frame}[blank]{The Microscope}
  \vspace{-0.5em}
  \includegraphics[width=\textwidth]{temVsoptical.pdf}
\end{frame}

\begin{frame}[fragile]{The Sample}
  \begin{center}\includegraphics[width=0.6\textwidth]{grid.png}
    \end{center}

    \begin{itemize}
    \item Sample must be thin so electrons can pass through AND to rapidly freeze.
    \item Ice must be vitreous (amorphic). Crystalline ice interacts strongly with electrons.
    \end{itemize}

    \tiny Costa, Tiago \& Ignatiou, Athanasios \& Orlova, E.V.. (2017). Structural Analysis of Protein Complexes by Cryo Electron Microscopy. 10.1007/978-1-4939-7033-9\_28. 
\end{frame}

\begin{frame}[fragile]{The Images}
  \vspace{-1.2em}
  \begin{center}\includegraphics[width=0.625\textwidth, height=0.625\textwidth]{betagal_good.png}
    \end{center}
\end{frame}
\begin{frame}[fragile]{The Images (More Realistic)}
  \vspace{-1.2em}
  \begin{center}\includegraphics[width=0.625\textwidth, height=0.625\textwidth]{10281_hard.png}
    \end{center}
\end{frame}

\section{Single Particle Analysis}
\subsection{}
\begin{frame}[fragile]{The Data}

  \placetextbox[north west]{0.03}{0.75}{\setlength{\fboxsep}{0pt}\setlength{\fboxrule}{0.5pt}\fbox{\includegraphics[width=0.3\textwidth, height=0.3\textwidth]{10281_hard.png}}}
  \placetextbox[north west]{0.04}{0.73}{\setlength{\fboxsep}{0pt}\setlength{\fboxrule}{0.5pt}\fbox{\includegraphics[width=0.3\textwidth, height=0.3\textwidth]{10281_hard.png}}}
  \placetextbox[north west]{0.05}{0.71}{\setlength{\fboxsep}{0pt}\setlength{\fboxrule}{0.5pt}\fbox{\includegraphics[width=0.3\textwidth, height=0.3\textwidth]{10281_hard.png}}}
  \placetextbox[north west]{0.06}{0.69}{\setlength{\fboxsep}{0pt}\setlength{\fboxrule}{0.5pt}\fbox{\includegraphics[width=0.3\textwidth, height=0.3\textwidth]{10281_hard.png}}}
  \placetextbox[north west]{0.07}{0.67}{\setlength{\fboxsep}{0pt}\setlength{\fboxrule}{0.5pt}\fbox{\includegraphics[width=0.3\textwidth, height=0.3\textwidth]{10281_hard.png}}}

  \placetextbox[north west]{0.40}{0.72}{\parbox{0.6\textwidth}{3D reconstruction can be made from many (many) projections of single particles at different orientations.}}
  \placetextbox[north west]{0.40}{0.50}{\parbox{0.6\textwidth}{If different orientations are present then angular information can be retrieved.}}

  \placetextbox[north west]{0.37}{0.33}{\setlength{\fboxsep}{0pt}\setlength{\fboxrule}{0.5pt}\fbox{\includegraphics[width=0.2\textwidth, height=0.2\textwidth]{repro_1.png}}}
  \placetextbox[north west]{0.57}{0.33}{\setlength{\fboxsep}{0pt}\setlength{\fboxrule}{0.5pt}\fbox{\includegraphics[width=0.2\textwidth, height=0.2\textwidth]{repro_2.png}}}
  \placetextbox[north west]{0.77}{0.33}{\setlength{\fboxsep}{0pt}\setlength{\fboxrule}{0.5pt}\fbox{\includegraphics[width=0.2\textwidth, height=0.2\textwidth]{repro_3.png}}}

\end{frame}



\begin{frame}[fragile]{The Pipeline relion\_it}
  

  \placetextbox[north west]{0.07}{0.77}{\includegraphics[width=0.35\textwidth]{relion.png}}
  \placetextbox[north west]{0.54}{0.77}{\textit{From Movies to Model}}
  \placetextbox[north west]{0.083}{0.30}{\movie[width=3.5cm,height=2cm,poster, externalviewer, loop]{\includegraphics[width=3.5cm]{movie_still.png}}{movie.webm}}
  \placetextbox[north west]{0.41}{0.72}{\includegraphics[width=0.6\textwidth]{pipeline.png}}
  \only<2>{\begin{tikzpicture}[remember picture,overlay]
    \node[xshift=71.8mm,yshift=-43.25mm,anchor=north west] at (current page.north west){%
    \includegraphics[width=22mm, height=8mm]{circle.png}};
\end{tikzpicture}}
\end{frame}

\section{My Project}
\subsection*{}
\begin{frame}[fragile]{Particle Picking}
 \large Ways to pick objects from noisy data \\
 \vspace{0.2em}
 \normalsize
  \begin{itemize}
    \item Laplacian of Gaussian
    \begin{itemize}
        \item[--] ``Blob Detector'' used in relion\_it
    \end{itemize}
    \item Template Matching
    \begin{itemize}
        \item[--] Bias to chosen 2D/3D classes
    \end{itemize}
    \item Deep Learning
    \begin{itemize}
        \item[--] Difficult to generalise but more selective!
    \end{itemize}
  \end{itemize}

  \begin{center}
    \begin{figure}
      \tiny
    \includegraphics[width=0.68\textwidth]{yolo.png}
    \caption{YOLO Convolutional Architecture}
    \end{figure}
    \tiny YOLOv3: An Incremental Improvement. Redmon, Josheph and Farhadi, Ali, 2018
    \placetextbox[north west]{0.675}{0.77}{\includegraphics[width=0.37\textwidth, height=0.37\textwidth]{prot_blank.png}}
    \only<2>{\placetextbox[north west]{0.675}{0.77}{\includegraphics[width=0.37\textwidth, height=0.37\textwidth]{prot_cry.png}}}
  \end{center}

\end{frame}

\begin{frame}[fragile]{crYOLO with General Model}
    \only<1>{\includegraphics[width=0.50\textwidth]{prot_blank.png}}
    \only<1>{\includegraphics[width=0.50\textwidth]{prot_blank.png}}
    \only<2>{\includegraphics[width=0.50\textwidth]{prot_rel.png}}
    \only<2>{\includegraphics[width=0.50\textwidth]{prot_cry.png}}
    \placetextbox[north west]{0.25}{0.15}{LoG}
    \placetextbox[north west]{0.66}{0.15}{crYOLO}
\end{frame}

\begin{frame}[fragile]{On The Fly}
  \textit{Automated data processing during collection}
  \begin{itemize}
    \item Implemented crYOLO instead of LoG as picker for relion\_it OTF 
    \begin{itemize}\item[--] More accurate. Leads to good classes faster \end{itemize}
    \item Updated relion\_it to python 3 as well as adding features requested
    \item Packaged OTF script and available on pypi
    \item New OTF data collection has now been run on many sessions (\textgreater80) 
    \begin{itemize}\item[--] Many examples of crYOLO to assess performance \end{itemize}
  \end{itemize}
\end{frame}

\begin{frame}[fragile]{Current and Planned Work}
  \begin{itemize}
    \item Looking at possible improvements to when crYOLO performed badly
    \item Establishing a retraining loop for crYOLO model
    \item Testing how prefiltering data affects crYOLO 
    \item Exploring picking and reconstruction from tilt pairs
    \item Exploring picking from tomography data
  \end{itemize}
  \movie[width=2cm,height=2cm,poster, externalviewer, loop]{\includegraphics[width=2cm]{tilt_still.png}}{tilt.gif} \\ 
  \tiny Martin Pilhofer and Grant Jensen/HHMI/Janelia Research Campus
\end{frame}

\section{Training}
\subsection*{}
\begin{frame}[fragile]{Programming}
  \textbf{Python}
  \begin{itemize}
    \item Programming guidance from co-supervisor
    \item Learnt how to package a python program (relion-yolo-it on pypi)
    \item Completed a 3 day course on Python for Data Analysis 
  \end{itemize}
  \vspace{1em}

  \textbf{Machine Learning}
  \begin{itemize}
    \item Attended ML for cryo-EM workshop in September. Another in March
    \item Fortnightly meetings with CCPEM/SciML to discuss ML
    \item Attend SciML lectures at Harwell
  \end{itemize}
\end{frame}

\begin{frame}[fragile]{Cryo-EM}
  \begin{itemize}
    \item Completed a tutorial on Relion reaching a 3~\AA~resolution on Beta-gal
    \item Working understanding of cryo-EM workflow and software used to analyse data
    \item Lots of time spent manipulating cryo-EM data
    \item Began training in sample preparation and microscope operation
  \end{itemize}
  \placetextbox[north west]{0.36}{0.35}{\movie[width=3.5cm,height=2cm,poster, externalviewer, loop]{\includegraphics[width=3.5cm]{movie_still.png}}{movie.webm}}
\end{frame}

\section{University}
\subsection{}

\begin{frame}{University and Placement}
\begin{columns}
\begin{column}{.5\textwidth}
\liketitle{\only<2>{\transparent{0.4}}~~What I brought}
  \begin{itemize}
  \only<2>{\transparent{0.4}}
    \vspace{1.2em}
    \item[\only<2>{\transparent{0.4}}{\textbullet}] Knowing how to learn independently
    \vspace{1em}
    \item[\only<2>{\transparent{0.4}}{\textbullet}] Problem solving
    \vspace{1em}
    \item[\only<2>{\transparent{0.4}}{\textbullet}] Critical reading of theory
    \vspace{1em}
    \item[\only<2>{\transparent{0.4}}{\textbullet}] Fourier analysis
  \end{itemize} 
\end{column}
\begin{column}{.5\textwidth}
\liketitle{\only<1>{\transparent{0.4}}~What I will take back}
  \begin{itemize}
\only<1>{\transparent{0.4}}
    \vspace{0.7em}
    \item[\only<1>{\transparent{0.4}}{\textbullet}] Greatly improved programming skill and knowledge
    \vspace{0.7em}
    \item[\only<1>{\transparent{0.4}}{\textbullet}] Knowledge and experience in applying ML
    \vspace{0.7em}
    \item[\only<1>{\transparent{0.4}}{\textbullet}] Understanding of how research is conducted
    \vspace{0.7em}
    \item[\only<1>{\transparent{0.4}}{\textbullet}] Data Analysis techniques!
  \end{itemize} 
\end{column}
\end{columns}
\end{frame}

\end{document}